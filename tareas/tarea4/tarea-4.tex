\documentclass[10pt]{article}
\usepackage{logo}
\usepackage{fullpage}
\usepackage[latin1]{inputenc}
\usepackage[spanish]{babel}
\usepackage{epsfig}
\usepackage{amsmath}
\usepackage{amssymb}
\usepackage{multicol}
\usepackage{color}

\newcommand{\cyan}[1]{\textCyan #1\textBlack}
\newcommand{\red}[1]{\textRed #1\textBlack}
\newcommand{\green}[1]{\textGreen #1\textBlack}
\newcommand{\blue}[1]{\textBlue #1\textBlack}
\newcommand{\black}[1]{\textBlack #1\textBlack}
\newcommand{\magenta}[1]{\textMagenta #1\textBlack}
\newcommand{\brown}[1]{\textBrown #1\textBlack}
\newcommand{\vs}[1]{\vspace{#1mm}}
\newcommand{\ignore}{}
\newcommand{\ri}[1]{\text{\red{#1}}}
\newcommand{\hs}{\hat\sigma}
\newcommand{\modelos}{{\it modelos}}
\newcommand{\B}{{\tt B}}
\newcommand{\nspace}{\text{NSPACE}}
\newcommand{\logspace}{\text{LOGSPACE}}
\newcommand{\nlogspace}{\text{NLOGSPACE}}
\newcommand{\npspace}{\text{NPSPACE}}
\newcommand{\crp}{\text{RP}}
\newcommand{\bpp}{\text{BPP}}
\newcommand{\pspace}{\text{PSPACE}}
\newcommand{\ph}{\text{PH}}
\newcommand{\expspace}{\text{EXPSPACE}}
\newcommand{\nexpspace}{\text{NEXPSPACE}}
\newcommand{\dspace}{\text{DSPACE}}
\newcommand{\espacio}{{\it espacio}}
\newcommand{\tiempo}{{\it tiempo}}
\newcommand{\ptime}{\text{P}}
\newcommand{\dtime}{\text{DTIME}}
\newcommand{\exptime}{\text{EXPTIME}}
\newcommand{\nexptime}{\text{NEXPTIME}}
\newcommand{\CC}{{\cal C}}
\newcommand{\sat}{\text{SAT}}
\newcommand{\tcnfu}{\text{3-CNF-SAT-UNSAT}}
\newcommand{\usat}{\text{unique-SAT}}
\newcommand{\np}{\text{NP}}
\newcommand{\ntime}{\text{NTIME}}
\newcommand{\cnf}{\text{CNF-SAT}}
\newcommand{\tcnf}{\text{3-CNF-SAT}}
\newcommand{\stcnf}{\#\text{3-CNF-SAT}}
\newcommand{\dcnf}{\text{2-CNF-SAT}}
\newcommand{\horn}{\text{HORN-SAT}}
\newcommand{\nhorn}{\text{NEG-HORN-SAT}}
\newcommand{\co}{\textssecl{co-}}
\newcommand{\rp}{\leq^\text{\it p}_\text{\it m}}
\newcommand{\tur}{\leq^\text{\it p}_\text{\it T}}
\newcommand{\reach}{\text{CAMINO}}
\newcommand{\pe}{\text{PROG-ENT}}
\newcommand{\pl}{\text{PROG-LIN}}
\newcommand{\cor}{\text{CONT-REG}}
\newcommand{\er}{\text{EQUIV-REG}}
\newcommand{\qbf}{\text{QBF}}
\newcommand{\hp}{\text{HP}}
\newcommand{\cdp}{\text{DP}}
\newcommand{\clique}{\text{CLIQUE}}
\newcommand{\seclique}{\#\text{EXACT-CLIQUE}}
\newcommand{\eclique}{\text{exact-CLIQUE}}
\newcommand{\costo}{\text{costo}}
\newcommand{\tsp}{\text{TSP}}
\newcommand{\tspu}{\text{unique-TSP}}
\newcommand{\no}{\text{NO}}
\newcommand{\yes}{\text{YES}}
\newcommand{\br}{\text{CERTAIN-ANSWERS}}
\newcommand{\dpc}{\text{DP}}

\newcommand{\CROM}{\text{CROM}}
\newcommand{\EVAL}{\text{EVAL}}
\newcommand{\EQUIV}{\text{EQUIV}}

\newcommand{\re}{\text{RE}}
\newcommand{\dec}{\text{R}}
\newcommand{\resp}[1]{{\text{\bf [Responsable: #1]}}}
\newcommand{\rpar}{\leq^\text{\it p}_\text{\it par}}
\newcommand{\pr}{\text{\rm {\bf Pr}}}
\newcommand{\cA}{\mathcal{A}}
\newcommand{\cB}{\mathcal{B}}

\newcommand{\minr}{\text{Min}}
\newcommand{\maxr}{\text{Max}}
\newcommand{\exir}{\text{Exists}}
\newcommand{\shp}{\text{\#P}}
\newcommand{\up}{\text{UP}}
\newcommand{\pg}{\text{GP}}


\begin{document}


\begin{tabular}{ccl}
\begin{tabular}{c}
\psfig{file=puclogo.eps}
\end{tabular}
&\ \ \ & 
\begin{tabular}{l}
PONTIFICIA UNIVERSIDAD CATOLICA DE CHILE\\
ESCUELA DE INGENIERIA\\
DEPARTAMENTO DE CIENCIA DE LA COMPUTACION
\end{tabular}
\end{tabular}

\vspace{1cm}

\begin{center}
\bf T�picos Avanzados en Teor�a de la Computaci�n - IIC3810\\
\bf Tarea 4\\
\bf Fecha de entrega: Martes 8 de octubre
\end{center}

\vspace{1cm}

Responda dos de las siguientes tres preguntas.
\begin{enumerate}
\item Construya una $p$-relacion $R$ tal que $\exir(R)$ es $\np$-completo y $f_R$ no es $\shp$-completo bajo reducciones parsimoniosas.


\item Sea $R_\pg$ la siguiente $p$-relaci�n:
  \begin{eqnarray*}
    R_\pg &=& \{ (G, C) \mid G \text{ es un grafo plano y } C \in \{1,2,3,4\}^* \text{ es una 4-coloraci�n de G}\}.
  \end{eqnarray*}
  En particular, si el conjunto de nodos de $G$ es $\{1, \ldots, n\}$,
  entonces $C = c_1 \cdots c_n$, donde $c_i \in \{1, 2, 3, 4\}$ indica
  el color del nodo $i$ para cada $i \in \{1, \ldots, n\}$. N�tese que
  $\exir(R_\pg) \in \ptime$ puesto que un grafo plano
  siempre puede ser pintado con 4 colores. Demuestre que la relaci�n
  $R_\pg$ no es auto-reducible.

\item Dada una clase de complejidad $\mathcal{C}$ de problemas de
  decisi�n, considera la definici�n de la clase de complejidad $\#
  \cdot \mathcal{C}$ de problemas de conteo dada en el siguiente art�culo:
  \begin{itemize}
  \item Lane A. Hemaspaandra, Heribert Vollmer: The satanic notations: counting classes beyond $\shp$ and other definitional adventures. SIGACT News 26(1):2--13 (1995).
    \end{itemize}
  En particular, bajo esta definici�n se tiene que $\shp = \# \cdot
  \ptime$ y $\shp \subseteq \# \cdot \np$.

  Adem�s, considere la clase
  $\up$ de problemas de decisi�n tal que un lenguaje $L$ sobre un
  alfabeto $\Sigma$ est� en $\up$ si y s�lo si existe una m�quina de
  Turing no determinista $M$ que funciona en tiempo polinomial tal que
  para todo $x \in \Sigma^*$:
  \begin{itemize}
  \item Si $x \in L$, entonces existe exactamente una ejecuci�n de $M$ con entrada $x$ que se detiene en un estado final.
  \item Si $x \not\in L$, entonces no existe una ejecuci�n de $M$ con entrada $x$ que se detiene en un estado final.
  \end{itemize}
  Vale decir, $M$ acepta los elementos de $L$ de manera no
  ambigua. Claramente se tiene que $\up \subseteq \np$, y se conjetura
  que $\up$ est� contenida en forma propia en $\np$.

  En esta pregunta usted debe demostrar que $\shp = \# \cdot \np$ si y s�lo si $\up = \np$.

\end{enumerate}

\end{document}
